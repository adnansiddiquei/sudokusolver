%! Author = adnansiddiquei
%! Date = 08/12/2023

\section{Documentation, Packaging and Usability}\label{sec:documentation-packaging-and-usability}
    \subsection{Documentation - \inlinecode{sphinx} and NumPy Style Docstrings}\label{subsec:documentation-sphinx}
    Documentation is an important part of any software project, without it, the user would have no idea how to utilise
    third-party packages.
    Documentation comes in two forms: inline documentation (docstrings and comments) and usage documentation (user guides
    and API references).
    The former is written whilst developing the code, and the latter is generated from the former with documentation tools.

    The discussion of which documentation tool to use is quite an opinionated one, there is no strict "one is better than
    the other" in most cases.
    The two most popular ones in python are \inlinecode{sphinx} and \inlinecode{doxygen}.
    Whilst using either of these tools would have been sufficient, \inlinecode{sphinx} was chosen for this project
    primarily because of the vast universe of themes and extensions available for it.
    The reason for documentation is to help a user understand how to use the package, and as such, above all else
    the documentation output should be easy to read and navigate.
    \inlinecode{sphinx} has a number of themes available, and the one chosen for this project was \inlinecode{furo} which
    is used by \inlinecode{pip} \cite{pip-docs}\, \inlinecode{black} \cite{black-docs} and the python developer's guide
    \cite{python-devguide}.
    Whilst the setup for \inlinecode{sphinx} was a bit more hands on, it resulted in much more usable and aesthetic docs.

    Another motivator for using \inlinecode{sphinx} was the ability to use NumPy style docstrings.
    More so than the desire to use NumPy style docstrings, was the desire to not use Epytext style or reStructuredText style
    docstrings, which is what Doxygen requires.
    We found these styles to be much less readable than NumPy style docstrings, which was another motivator for using
    \inlinecode{sphinx}.
    Another popular alternative is Google style doctrings, NumPy style was chosen only by personal preference, \inlinecode{sphinx}
    can handle both.

    \subsection{Packaging - Conda and Docker [TODO]}\label{subsec:packaging-pypi}
    What benefits does conda have over pip? Why did I choose conda?
    Why Docker?