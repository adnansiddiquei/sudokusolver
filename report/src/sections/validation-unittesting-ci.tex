%! Author = adnansiddiquei
%! Date = 12/12/2023

\section{Validation, Unit Tests and CI set up}\label{sec:validation-unit-tests-and-ci-set-up}
    \subsection{Unit Tests}\label{subsec:validation}
    Unit tests were implemented using \inlinecode{pytest}.
    These were planned and written prior to writing any code, as discussed in Section\eqref{subsec:test-driven-development}.
    All the functions described in the pseudocode in Fig.\eqref{fig:pseudocode} were tested, as well as the python wrapper
    for the \inlinecode{has_non_zero_duplicates} C function, and generally speaking, the tests covered the following cases:
    \begin{itemize}
        \item correct outputs when the inputs were correct;
        \item expected behaviour and handling of exceptions when inputs were incorrect.
        A function could take parameters that were incorrect in multiple ways (value, shape, length, type, etc.), and as such,
        we tested for all of these cases, as well as different combinations of these cases.
        \item correct handling of edge cases, where inputs were at their extremes (such as empty lists, input
        sudoku already solved, etc.)
        \item correct handling of slight variations on inputs that were allowed to be correct.
        For example, in the case of the input file, we allowed slightly incorrectly formatted versions of the input file
        to be parsed correctly.
    \end{itemize}

    Designing the tests using this framework, we were confident that the code was working as expected and that the testing
    suite was comprehensive.

    \subsection{Continuous Integration - \inlinecode{pre-commit}}\label{subsec:continuous-integration}
    \inlinecode{pre-commit} was used to set up actions to run before every commit.
    In our case, we used \inlinecode{pre-commit} to run the \inlinecode{ruff} linter and formatter before every commit.
    This ensured that no incorrectly formatted code made it's way into main.
    It is good to also run tests periodically, some projects run tests before every commit, and some run tests before
    every PR into main by using a CI tool such as GitHub Actions.
    However, in our case, we decided not to add the running of unit tests before every commit to
    \inlinecode{pre-commit-config.yaml} as we were doing test driven development, and as such, on many commits
    and PRs into main, the tests would, by design, not pass.
    However, we ensured that the tests were run regularly and as the relevant code was being written.
