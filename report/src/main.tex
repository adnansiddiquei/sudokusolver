%! Author = adnansiddiquei
%! Date = 07/12/2023

% Preamble
\documentclass[a4paper,11pt]{article}
\pdfoutput=1

% Packages
\usepackage{jcappub}
\usepackage[T1]{fontenc}
\usepackage{listings}
\usepackage{roboto}

\newcommand{\inlinecode}[1]{\lstinline{#1}}
\lstset{basicstyle=\fontfamily{pcr}\selectfont}


\title{\boldmath An implementation of a python sudoku solver package and a complete review of the software development
process involved.}


% %simple case: 2 authors, same institution
 \author{Adnan Siddiquei}
 \affiliation{University of Cambridge}

% e-mail addresses: one for each author, in the same order as the authors
\emailAdd{as3438@cam.ac.uk}




%\abstract{Abstract...}



\begin{document}
\maketitle
\flushbottom

\section{Introduction}\label{sec:intro}

%! Author = adnansiddiquei
%! Date = 07/12/2023

\section{Solution Design}\label{sec:solution-design}
    \subsection{Selection of Solution Algorithm [TODO]}\label{subsec:solution-algorithm}
    Why did I choose backtracking?

    \subsection{Prototyping}\label{subsec:prototyping}
    Prior to writing any code, we prototyped the solution.
    Prototyping prior to coding allowed us to
    \begin{itemize}
        \item identify possible bugs and complexity earlier on in the development process, such that we could consider them
        prior to coding, rather than discover them after the fact;
        \item identify non-trivial and edge cases that might need to be considered while writing code and unit tests;
        \item identify API interfaces of the package, allowing us to write the unit tests beforehand (more on this when
        we talk about test driven development in Section\eqref{subsec:test-driven-development}).
        \item explore different implementations of the backtracking algorithm;
        \item identify python packages and resources that we may need to use in the implementation, and make decisions on which
        might be the most appropriate;
    \end{itemize}
    The implementation of the sudoku solver consisted of two main components: parsing the user input (and handling
    associated errors), and the backtracking algorithm itself.
    The prototyping flowcharts of these two components is shown in Fig.\eqref{fig:input-prototype} and Fig.\eqref{fig:backtracking-prototype}.
    Additionally, we wrote some \textit{pseudo-python-code} as shown in Fig.\eqref{fig:pseudocode}, allowing us to define
    the interfaces of the functions and classes that we would write - which is required to write any unit tests before prior
    to coding.
    The tests derived from this solution design are discussed more in Section\eqref{sec:validation-unit-tests-and-ci-set-up}.
    Further iterations on these initial prototypes are discussed more in Section\eqref{sec:development-experimentation-and-profiling}
    where we discuss how experimentation and profiling changed the end implementation in efforts to optimise the code.

    \subsection{API Interface}\label{subsec:api-interface}
    An important decision to make early on is how the code will be structured into modules and what the external
    API interface will look like, as shown in Fig.\eqref{fig:pseudocode}.
    We decided to model the \inlinecode{sudokusolver} package after well known \inlinecode{scikit-learn} \cite{scikit-repo}
    package because \inlinecode{scikit-learn}'s structure is extremely well thought out and simple to use.
    Each solver would be segmented into own module, containing a class which can be instantiated with the hyperparameters
    of the solver (in this case, is there \inlinecode{multiple_solutions}), and then a \inlinecode{solve} method which
    takes in the data (the \inlinecode{unsolved_board}) and returns the class instance with the solved board as an attribute.
    This seemed like a fitting way to model the package as our solvers are akin to \inlinecode{scikit-learn}'s estimators,
    and the \inlinecode{solve} method is akin to \inlinecode{scikit-learn}'s \inlinecode{fit} method.
    To extend the package we simply need to add more solvers in their own modules, and to extend a solver we simply need
    to add more hyperparameters to the class and modify the \inlinecode{solve} method to take these hyperparameters into
    account.

    \subsection{Key Conclusions}\label{subsec:key-conclusions}
    We decided to only implement the \inlinecode{BacktrackingSolver} with no \inlinecode{multiple_solutions} functionality
    in the initial \inlinecode{v0.0.1} implementation of the package,
    The implementation of \inlinecode{multiple_solutions} was not thought through in the prototyping stage but given that
    we now understood how to syntactically and structurally extend the package and the solvers within, it would be trivial
    to add this functionality in a future version of the package, which demonstrates the importance of blueprinting out
    the API interface.

    Another key decision which was implicit and not thoroughly discussed was to utilise \inlinecode{numpy} arrays to
    represent the sudoku board, this is because \inlinecode{numpy} is well known to be the most performant array manipulation
    package in python.
    Alternatives include using \inlinecode{pandas} or python's in-built \inlinecode{list} object, however, \inlinecode{numpy}
    is generally known to be more performant than both of these and \inlinecode{numpy} also includes numerous built-in functions
    that will likely be useful in the implementation of the backtracking algorithm.

    \begin{figure}[htb]
    \centering
    \includegraphics[width=0.9\textwidth]{./figures/parse-input-prototype}
    \caption{A flowchart for part 1 of solving a sudoku puzzle: parsing the user input.}
    \label{fig:input-prototype}
    \end{figure}

    \begin{figure}[htb]
    \centering
    \includegraphics[width=0.9\textwidth]{./figures/backtracking-prototype}
    \caption{A flowchart for part 2 of solving a sudoku puzzle: the backtracking algorithm.}
    \label{fig:backtracking-prototype}
    \end{figure}

    \begin{figure}[htb]
    \centering
    \begin{lstlisting}[language=Python,label={lst:lstlisting}]
        class BacktrackingSolver
            def __init__(self, multiple_solutions: bool = False):
                self.board  # will store solved board
                self.is_solvable  # will store whether board is solvable
                self.is_solved  # will store whether board is solved
            def solve(self, unsolved_board: numpy.NDArray) -> None:

        def parse_input_file(input_file_path: str) -> numpy.NDArray

        def save_board(
            board: numpy.NDArray,
            output_file_path: str
        ) -> None

        # entry point for script, argv = sys.argv
        def handler(argv) -> None:
    \end{lstlisting}
    \caption{\textit{pseudo-python-code} representation of the interfaces of the functions and classes within our
    \inlinecode{sudokusolver} package.}
    \label{fig:pseudocode}
    \end{figure}


%! Author = adnansiddiquei
%! Date = 07/12/2023

\section{Development, Experimentation and Profiling}\label{sec:development-experimentation-and-profiling}
Here we discuss several components of the development process, and reason why we chose to do things in certain ways

\subsection{Linting and Formatting - \inlinecode{ruff}}\label{subsec:linting-and-formatting}
    Linting and formatting is useful, especially in shared projects, as it allows for a consistent style across the codebase.
    There are a plethora of python linting and formatting tools available and for this project, we chose to use \inlinecode{ruff}.
    There were two primary reasons for this choice: speed and simplicity.
    \inlinecode{ruff} is faster than most other linting tools including \inlinecode{flake8}.
    In a test done by the developers of \inlinecode{ruff}, it managed to lint the CPython codebase 42x faster than
    \inlinecode{flake8} \cite{ruff-repo}.
    Whilst speed of linting is not a primary concern for this project, it doesn't hurt to pick the faster option.

    Additionally, \inlinecode{ruff} provides formatting functionality and as such it can also replace tools such as
    \inlinecode{black}.
    This makes the implementation of linting and formatting simpler, as we only need to use one tool.
    \inlinecode{ruff}'s configuration capabilities allow it to lint and format to any standard we want to, and as such,
    it was configured to mimic \inlinecode{black} and \inlinecode{flake8}'s default config in accordance with PEP8.

    \subsection{Git and Gitlab pipeline}\label{subsec:git-and-gitlab-pipeline}
    Why branches? Branch naming convention.
    Creating issues for tasks. Why? Project management. JIRA.
    New branch and merge per issue.
    Labels for issues because easy to and can folder the branches with those labels (show PyCharm example).

    \subsection{Test Driven Development}\label{subsec:test-driven-development}
    Explain why I wrote the tests first.

    \subsection{Profiling and Optimisation}\label{subsec:profiling-and-optimisation}
    Where did i test different packages for speed? Why did I choose numpy?
    Show how I profiled my code to identify where the bottleneck was.
    How did I work around this? Cython?

    \subsection{Coding Best Practises}\label{subsec:coding-best-practises}
    Modularisation.
    typing.
    Exceptions. Error handling, try except. Never catch all exceptions.


\section{Validation, Unit Tests and CI set up}\label{sec:validation-unit-tests-and-ci-set-up}
Talk through why my unit tests are sufficient.
How did I put this into the CI? With pre-commit.yaml


\section{Documentation, Packaging and Usability}\label{sec:documentation-packaging-and-usability}
\subsection{Documentation - \inlinecode{sphinx}}\label{subsec:documentation-sphinx}
Why did I use sphinx over Doxygen? Sphinx is more popular, more support, more documentation, more features.

\subsection{Packaging - Conda and Docker}\label{subsec:packaging-pypi}
What benefits does conda have over pip? Why did I choose conda?
Why Docker?

\section{Summary}
\label{sec:conclusion}


\begin{thebibliography}{99}

\bibitem{ruff-repo}
Astral,
\textit{\inlinecode{ruff} GitHub Repository}.
Available at: \url{https://github.com/astral-sh/ruff}
[Accessed: 7-Dec-2023].

\bibitem{scikit-repo}
scikit-learn,
\textit{\inlinecode{schikit-learn} GitHub Repository}.
Available at: \url{https://github.com/scikit-learn/scikit-learn}
[Accessed: 7-Dec-2023].

\end{thebibliography}



\end{document}
